\item {\bf Value Iteration on Mountain Car}

Now that we have gotten a bit of practice with general-purpose MDP algorithms, let's use them for some control problems.
Mountain Car is a classic example in robot control \cite{fn-7} where you try to get a car to the goal located on the top of a steep hill by accelerating left or right.
We will use the implementation provided by The Farama Foundation's Gymnasium, formerly OpenAI Gym.

The state of the environment is provided as a pair (position, velocity).
The starting position is randomized within a small range at the
bottom of the hill. At each step, the actions are either to accelerate to the left, to the right, or do nothing, and the transitions are
determined directly from the physics of a frictionless car on a hill. Every step produces a small negative reward, and reaching the goal provides a large
positive reward.

To get the feel for the environment, test with an untrained agent which takes random action at each step and see how it performs.

\texttt{python mountaincar.py --agent naive}

You will see the agent struggling, not able to complete the task within the time limit. 
In this assignment, you will train this agent with different reinforcement learning algorithms so that it can learn to climb the hill.
As the first step, we have designed two MDPs for this task. The first uses the car's continuous (position, velocity) state as is, and the second discretizes
the position and velocity into bins and uses indicator vectors.

Carefully examine |ContinuousGymMDP| and |DiscreteGymMDP| classes in |util.py| and make sure you understand.

If we want to apply value iteration to the |DiscreteGymMDP| (think about why we can't apply it to ContinuousGymMDP), 
we require the transition probabilities $T(s, a, s')$ and rewards $R(s, a, s')$ to be known. But oftentimes in the real world, $T$ and $R$ are unknown, 
and the gym environments are set up in this way as well, only interfacing through the |.step()| function. One method
to still determine the optimal policy is model-based value iteration, which runs Monte Carlo simulations to estimate $\hat{T}$ and $\hat{R}$, and then runs value iteration. 
This is an example of model-based RL. Examine |RLAlgorithm| in |util.py| to understand the |getAction| and |incorporateFeedback| interface 
and peek into the |simulate| function to see how they are called repeatedly when training over episodes.

\begin{enumerate}

  \item \points{3a}

As a warm up, we will start with implementing value iteration as you learned in lectures, and run it on the number line MDP from Problem 1. 

Complete |valueIteration| function in |submission.py|.

  \item \points{3b}

Now in |submission.py|, implement |ModelBasedMonteCarlo| which runs |valueIteration| 
every |calcValIterEvery| steps that the RL agent operates. The |getAction| method controls how we use the latest policy as determined by value iteration, 
and the |incorporateFeedback| method updates the transition and reward estimates, calling value iteration when needed.
Implement the |getAction| and |incorporateFeedback| methods for |ModelBasedMonteCarlo|.


  \input{03-value-iteration-mountain-car/03-simulation}

\end{enumerate}
