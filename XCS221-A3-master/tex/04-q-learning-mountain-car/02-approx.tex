\item \points{4b}

For Q-learning in continuous states, we need to use function approximation. The first step of function approximation is extracting 
features given state. Feature extractors of different complexities work well with different problems: linear and polynomial feature 
extractors that work well with simpler problems may not be suitable for other problems. For the mountain car task, we are going to use 
a Fourier feature extractor. As background, any continuous periodic function can be approximated as a Fourier Series
$$f(x) = \frac{a_0}{2} + \sum_{j=1}^n\left[a_j\cos(2\pi j x/T) + b_j\sin(2\pi j x/ T)\right]$$
with $a_j$ and $b_j$ sequences of coefficients determined by integrating $f$. To apply this to Q-learning with function approximation,
we want the learned weights $w$ to emulate $a_j$ and $b_j$ and the output of $\phi$ to provide the basis of varying sinusoid periods
as seen in $\cos(2\pi j x/T)$ for $j = 1, 2, \ldots, n$.
Thus, for state $s = [s_1, s_2, \ldots, s_k]$, action $a$, and maximum coefficient $c$, the feature extractor $\phi$ is:
$$\phi(s, a, c) = [\cos(0), \cos(s_1), \ldots, \cos(s_k), \cos(2s_1), \cos(s_1+s_2), \ldots, \cos(cs_1 + cs_2 + \ldots + cs_k)]$$
$$= \left\{\cos\left(\sum_{i=1}^k c_i s_i\right): \forall c_1\in\mathscr{C}, \ldots, c_k\in\mathscr{C}\right\}$$
where $\mathscr{C} = \{0, 1, \ldots, c\}$ is the set of non-negative integers from 0 to $c$. Note that $\phi(s, a, c) \in \mathbb{R}^{(c+1)^k}$.

Implement |fourierFeatureExtractor| in |submission.py|. Looking at |util.polynomialFeatureExtractor| may be useful
for familiarizing oneself with numpy.


